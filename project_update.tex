% Options for packages loaded elsewhere
\PassOptionsToPackage{unicode}{hyperref}
\PassOptionsToPackage{hyphens}{url}
\documentclass[
]{article}
\usepackage{xcolor}
\usepackage[margin=1in]{geometry}
\usepackage{amsmath,amssymb}
\setcounter{secnumdepth}{-\maxdimen} % remove section numbering
\usepackage{iftex}
\ifPDFTeX
  \usepackage[T1]{fontenc}
  \usepackage[utf8]{inputenc}
  \usepackage{textcomp} % provide euro and other symbols
\else % if luatex or xetex
  \usepackage{unicode-math} % this also loads fontspec
  \defaultfontfeatures{Scale=MatchLowercase}
  \defaultfontfeatures[\rmfamily]{Ligatures=TeX,Scale=1}
\fi
\usepackage{lmodern}
\ifPDFTeX\else
  % xetex/luatex font selection
\fi
% Use upquote if available, for straight quotes in verbatim environments
\IfFileExists{upquote.sty}{\usepackage{upquote}}{}
\IfFileExists{microtype.sty}{% use microtype if available
  \usepackage[]{microtype}
  \UseMicrotypeSet[protrusion]{basicmath} % disable protrusion for tt fonts
}{}
\makeatletter
\@ifundefined{KOMAClassName}{% if non-KOMA class
  \IfFileExists{parskip.sty}{%
    \usepackage{parskip}
  }{% else
    \setlength{\parindent}{0pt}
    \setlength{\parskip}{6pt plus 2pt minus 1pt}}
}{% if KOMA class
  \KOMAoptions{parskip=half}}
\makeatother
\usepackage{graphicx}
\makeatletter
\newsavebox\pandoc@box
\newcommand*\pandocbounded[1]{% scales image to fit in text height/width
  \sbox\pandoc@box{#1}%
  \Gscale@div\@tempa{\textheight}{\dimexpr\ht\pandoc@box+\dp\pandoc@box\relax}%
  \Gscale@div\@tempb{\linewidth}{\wd\pandoc@box}%
  \ifdim\@tempb\p@<\@tempa\p@\let\@tempa\@tempb\fi% select the smaller of both
  \ifdim\@tempa\p@<\p@\scalebox{\@tempa}{\usebox\pandoc@box}%
  \else\usebox{\pandoc@box}%
  \fi%
}
% Set default figure placement to htbp
\def\fps@figure{htbp}
\makeatother
\setlength{\emergencystretch}{3em} % prevent overfull lines
\providecommand{\tightlist}{%
  \setlength{\itemsep}{0pt}\setlength{\parskip}{0pt}}
\usepackage{bookmark}
\IfFileExists{xurl.sty}{\usepackage{xurl}}{} % add URL line breaks if available
\urlstyle{same}
\hypersetup{
  pdftitle={Project Progress Check-in: Tennis Match Outcome Forecasting},
  pdfauthor={Luke Feng, Sam English, Blake Bothmer},
  hidelinks,
  pdfcreator={LaTeX via pandoc}}

\title{Project Progress Check-in: Tennis Match Outcome Forecasting}
\author{Luke Feng, Sam English, Blake Bothmer}
\date{2025-11-14}

\begin{document}
\maketitle

\subsection{Data Collection Status}\label{data-collection-status}

We are building a model to forecast ATP tennis match outcomes using both
historical performance data and betting market information. At this
stage, we have collected and successfully loaded the two core datasets:

\begin{enumerate}
\def\labelenumi{\arabic{enumi}.}
\item
  \textbf{Tennis-Data.co.uk (2010--2024)}\\
  This dataset provides match-level information including: date,
  tournament, location, surface, round, winner/loser names, ATP
  rankings, set scores, and odds from several bookmakers (e.g.~Bet365).
  We have cleaned and combined the yearly files into a single dataset,
  filtered to completed matches, standardized player names, and
  constructed features such as normalized Bet365 implied win
  probabilities and rank differences (loser rank -- winner rank).
\item
  \textbf{Jeff Sackmann's ATP match data (2000--2024)}\\
  This dataset contains full historical ATP match results, including
  player names, tournament dates, surfaces, and match outcomes. Using
  this data, we computed tennis-specific Elo ratings (global and
  surface-specific), head-to-head (H2H) records between players, and
  recent form features such as number of wins in the last 5--10 matches.
\end{enumerate}

Because the two sources use different naming conventions and dates
(tournament start date vs.~match date), we performed a fuzzy match based
on shortened player names and date proximity (matching within ±3 days).
This gives us a merged dataset where about \textbf{60\%} of matches in
the Tennis-Data.co.uk file have corresponding Elo/H2H/form features from
the Sackmann data.

We estimate that we currently have about \textbf{65--70\%} of the data
we would ideally like for the final version of the project.
Specifically:

\begin{itemize}
\tightlist
\item
  Market odds + match outcomes: essentially 100\% for ATP matches
  2010--2024.
\item
  Elo, H2H, and recent form features: available for \textasciitilde60\%
  of those matches (the subset where we can confidently match players
  and dates).
\item
  Detailed ATP player statistics (serve \%, return \%, etc.): 0\%
  collected so far and planned as a later enrichment step.
\end{itemize}

Our next data-collection step is to scrape or otherwise obtain serve and
return (as well as other relevant) statistics from the ATP website or an
equivalent source, then merge those statistics into the existing
modeling dataset.

\subsection{Exploratory Data Analysis}\label{exploratory-data-analysis}

To verify that our data is usable and that we understand its structure,
we performed several exploratory analyses on the merged dataset (betting
+ Elo/H2H/form). A few key examples:

\begin{itemize}
\tightlist
\item
  \textbf{Distribution of implied win probabilities.}
\end{itemize}

\includegraphics[width=0.7\linewidth]{figures/distr_win_prob}

We computed normalized Bet365 implied probabilities for the eventual
winner. The distribution is right-skewed, with most winners having
pre-match probabilities between 0.6 and 0.8. A small left tail reflects
upset matches, consistent with tennis being a sport where favorites
typically win but surprises still occur.

\begin{itemize}
\tightlist
\item
  \textbf{Distribution of rank differences.}
\end{itemize}

\includegraphics[width=0.7\linewidth]{figures/distr_rank}

Using rank\_diff = lrank − wrank, we see that many matches involve a
winner who was higher-ranked (positive values). The left tail
corresponds to ranking upsets. This confirms that ATP ranking is
informative but imperfect.

\begin{itemize}
\tightlist
\item
  \textbf{Relationship between Elo difference and market-implied
  probability.}
\end{itemize}

\includegraphics[width=0.7\linewidth]{figures/elo_difference_market_prob}

This scatter plot shows a strong positive relationship: as global Elo
difference (winner Elo − loser Elo) increases, the betting market
assigns higher win probabilities to the winner. This suggests that our
Elo system is capturing similar skill differences to the bookmakers.

\begin{itemize}
\tightlist
\item
  \textbf{Surface distribution.}
\end{itemize}

\includegraphics[width=0.7\linewidth]{figures/match_counts_surface}

Hard courts account for the largest share of matches, followed by clay
and grass. This matches the ATP calendar and supports including surface
type and surface-specific Elo ratings as key predictors.

\subsection{Biggest Unresolved Issue}\label{biggest-unresolved-issue}

The single biggest unresolved issue is \textbf{integrating detailed ATP
player statistics (serve \%, return \%, break point stats etc.) into our
modeling dataset}.

Conceptually, we would like to include features such as:

\begin{itemize}
\tightlist
\item
  Career and season-level first serve percentage
\item
  First and second serve points won
\item
  Return points won
\item
  Break point conversion and save rates
\item
  Service and return games won
\end{itemize}

Data Challenges include:

\begin{enumerate}
\def\labelenumi{\arabic{enumi}.}
\tightlist
\item
  Accessing and parsing player statistics pages
\item
  Normalizing inconsistent player naming conventions
\item
  Aligning career/seasonal stats with specific match dates
\item
  Avoiding rate limits and errors during scraping
\end{enumerate}

Our plan for resolving this is to start from an existing scraping
repository (e.g., the \texttt{infotennis} project on GitHub that targets
ATP stats), adapt it to our list of players, and store the resulting
stats in a separate player- level table. We can then merge those
player-level features into the match- level modeling dataset by player
and year. If scraping proves to be too fragile, we will explore
alternative public datasets (such as Tennis Abstract CSVs or Kaggle
datasets with aggregated player stats).

\subsection{External Elements Beyond the
Course}\label{external-elements-beyond-the-course}

We plan to incorporate several elements beyond standard course material:

\begin{itemize}
\item
  \textbf{Sports analytics methods.}\\
  Custom Elo rating systems, surface adjustments, and dynamic K-factors
  inspired by Jeff Sackmann's tennis analytics work.
\item
  \textbf{Web scraping and data engineering.}\\
  Automated scraping of ATP statistics and merging player-level and
  match-level features.
\end{itemize}

\end{document}
